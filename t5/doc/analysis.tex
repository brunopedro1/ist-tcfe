\newpage
\epstopdfsetup{outdir=./}
\section{Theoretical Analysis}
\label{sec:analysis}

In this section we will theoretically analyse the circuit of Figure \ref{fig:t5} using OCTAVE tools. This analysis will start with the calculation of the output voltage gain and central frequency. Then, the input and output impedances at the central frequency previously calculated will be determined. Finally, a frequency response analysis is presented.


\subsection{Output Voltage Gain and Central Frequency}


To determine the central frequency, the angular upper ($\omega_H$) and lower ($\omega_L$) frequencies were calculated using the following expressions:

\begin {equation}
	\omega_H = \frac{1}{R_2 *C_2}  
	\label{eq:wh}
\end{equation}

\begin {equation}
	\omega_L= \frac{1}{R_1*C_1}   
	\label{eq:wl}
\end{equation}

The central frequency is now given by:

\begin {equation}
	central frequency = \frac{\sqrt{\omega_L * \omega_H }}{2\pi}
	\label{eq:CentralFreq}
\end{equation}

The output voltage gain, in dB, was obtained through the following expression

\begin {equation}
	Gain = 20*log(|\frac{R_1*C_1*\sqrt{\omega_L * \omega_H }*j}{1+R_1*C_1*\sqrt{\omega_L * \omega_H }*j}*(1+\frac{R_3}{R_4})*\frac{1}{1+R_2*C_2*\sqrt{\omega_L * \omega_H }*j}|)
	\label{eq:gain}
\end{equation} 

The values obtained are shown in Table \ref{tab:teoresults}.

\begin{table}[H]
	\centering
	\begin{tabular}{|l|r|}
		\hline    
		{\bf Name} & {\bf Value} \\ \hline
		\input{../mat/teoresults_TAB}
	\end{tabular}
	\caption{Theoretical output voltage gain and central frequency.}
	\label{tab:teoresults}
\end{table}

\subsection{Input and Output Impedances at Central Frequency}

The input impedance at the central frequency determined previously was calculated using the following expression:

\begin {equation}
	Z_{in} = |R_1 + \frac{1}{j*\sqrt{\omega_L * \omega_H }*C_1}|
	\label{eq:Zin}
\end{equation}  

And the output impedance at the central frequency was calculated using the following expression:

\begin {equation}
       Z_{out} = |\frac{1}{j*\sqrt{\omega_L * \omega_H }*C_2+\frac{1}{R2}}|
	\label{eq:Zout}
\end{equation} 


Its values are shown in Table \ref{tab:teoresults2}.

\begin{table}[H]
	\centering
	\begin{tabular}{|l|r|}
		\hline    
		{\bf Name} & {\bf Value} \\ \hline
		\input{../mat/teoresults2_TAB}
	\end{tabular}
	\caption{Theoretical input and output impedances at central frequency.}
	\label{tab:teoresults2}
\end{table}


\subsection{Frequency Response}

The graphic of Figure \ref{fig:teo_gain} shows the frequency response of the output voltage gain, in dB, of the BandPass Filter.

\begin{figure}[H] \centering
	\includegraphics[width=0.8\linewidth]{teo_gain.eps}
	\caption{Frequency response of the output voltage gain. The \textit{x axis} represents the frequency in Hertz (Hz) and the \textit{y axis} the Gain in dB.}
	\label{fig:teo_gain}
\end{figure}

The graphic of Figure \ref{fig:teo_phase} shows the frequency response of the phase, in degrees.

\begin{figure}[H] \centering
	\includegraphics[width=0.8\linewidth]{teo_phase.eps}
	\caption {Frequency response of the phase. The \textit{x axis} represents the frequency in Hertz (Hz) and the \textit{y axis} the Phase in degrees.}
	\label{fig:teo_phase}
\end{figure}




