\newpage
\section{Comparison}
\label{sec:comparison}
\par
We will now compare the results of more interest obtained in the theoretical analysis and the simulation analysis.\par


For the impedance values, the Theoretical Results were the following:



And the Simulation results obtained were:


 


In both cases the Output Impedance is under enough to prevent signal loss when connecting the circuit to a speaker of 8Ohm.


For the Gain Frequency Response this is the graphic obtained theoretically:


And the simulation one is:


Analyzing these values we notice that there is some difference between the theoretical predicion and the simulation. But we can also notice that, besides the differences, the results obtained were relatively good in both analysis. The simulation analysis gave the best results.

The differences seen in the results can be attributed to diverse causes. The differences between the values of impedance can be due to the fact that in the Simulation, these are not calculated for the intermediate stages of the circuit, while that was done in the theoretical analysis.\par
These differences also can be given to the fact that the circuit is quite complex, and contains non linear elements like transistors which always influences the accuracy of the results obtained, especially in the Simulation Analysis, so it explains the differences in the results obtained.\par


\newpage
\section{Conclusion}
\label{sec:conclusion}

Given the comparison made previously and the results obtained along the work, including a merit figure of more than one thousand, we can say that the objectives for this laboratory assignment were achieved and an Audio Amplifier circuit was built successfully. 





