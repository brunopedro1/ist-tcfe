\newpage
\epstopdfsetup{outdir=./}
\section{Simulation Analysis}
\label{sec:simulation}

~In this section, the circuit built (Figure \ref{fig:t5}) will be analyzed and simulated using NGSPICE. 

Before simulating the circuit shown before, we used NGSPICE tools to simulate different circuit configurations and combinations of values for the various components of the circuit to obtain a decent merit value, while trying to get the central frequency and gain asked. \par


\subsection{Output Voltage Gain and Central Frequency}

The gain in dB units is given in the Table \ref{tab:sim_gain}.

\begin{table}[H]
	\centering
	\begin{tabular}{|l|r|}
		\hline    
		{\bf Name} & {\bf Value [dB]} \\ \hline
		\input{../sim/gain_TAB}
	\end{tabular}
	\caption{Value of simulated output voltage gain.}
	\label{tab:sim_gain}
\end{table}

The Upper, Lower and Central Frequencies are given by the Table \ref{tab:sim_freq}. These values will be further discussed in the comparison section of this report, but just by observing we can tell these values are within the values expected, especially the central frequency value which is close to the expected 1kHz.

\begin{table}[H]
	\centering
	\begin{tabular}{|l|r|}
		\hline    
		{\bf Name} & {\bf Value [Hz]} \\ \hline
		\input{../sim/freq_TAB}
	\end{tabular}
	\caption{Simulated upper, lower and central frequencies}
	\label{tab:sim_freq}
\end{table}


\subsection{Input and Output Impedances at Central Frequency}

The Table \ref{tab:sim_zin} shows the Input Impedance at central frequency from simulation analysis.

\begin{table}[H]
	\centering
	\begin{tabular}{|l|r|}
		\hline    
		{\bf Name} & {\bf Value [kOhm]} \\ \hline
		\input{../sim/input_TAB}
	\end{tabular}
	\caption{Simulated Input Impedance at central frequency.}
	\label{tab:sim_zin}
\end{table}

The Table \ref{tab:sim_zout} shows the Output Impedance at central frequency from simulation analysis.

\begin{table}[H]
	\centering
	\begin{tabular}{|l|r|}
		\hline    
		{\bf Name} & {\bf Value [kOhm]} \\ \hline
		\input{../sim/output_TAB}
	\end{tabular}
	\caption{Simulated Output Impedance at central frequency.}
	\label{tab:sim_zout}
\end{table}


\subsection{Merit Figure}

To calculate the merit figure, the values of the output voltage gain deviation from 40dB, the central frequency deviation from 1kHz and the cost of the components used were calculated and can be seen in Table \ref{tab:sim_merit}, as well as the resulting merit figure.

\begin{table}[H]
	\centering
	\begin{tabular}{|l|r|}
		\hline    
		{\bf Name} & {\bf Value} \\ \hline
		\input{../sim/merit_TAB}
	\end{tabular}
	\caption{Values obtained for the merit figure calculation.}
	\label{tab:sim_merit}
\end{table}

\vspace{-2.5cm}
\subsection{Frequency Response}

The simulated output voltage gain in terms of frequency is shown in Figure \ref{fig:sim_gaindB}.
\vspace{-2cm}
\begin{figure}[H] \centering
	\includegraphics[width=0.8\linewidth]{vo1f.pdf}
	\caption {Frequency response of the gain in dB scale. The \textit{x axis} represents the frequency in Hertz (Hz) and the \textit{y axis} the gain in dB.}
	\label{fig:sim_gaindB}
\end{figure}

And the plot of the phase function is given by the graphic of Figure \ref{fig:sim_phase}.

\begin{figure}[H] \centering
	\includegraphics[width=0.8\linewidth]{phase.eps}
	\caption {Frequency response of the phase. The \textit{x axis} represents the frequency in Hertz (Hz) and the \textit{y axis} the phase in degrees.}
	\label{fig:sim_phase}
\end{figure}


