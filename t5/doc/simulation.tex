\newpage
\epstopdfsetup{outdir=./}
\section{Simulation Analysis}
\label{sec:simulation}

~In this section, the circuit built (Figure \ref{fig:audio_amplifier}) will be analyzed and simulated using NGSPICE. 

Before simulate the circuit shown before, we used NGSPICE tools to simulate different circuit configurations and combinations of values for the various components of the circuit to obtain a decent merit value, while trying to get good impedance values and gain response and a relatively low cost. \par


\subsection{Operating Point}

We calculated the Operating Point for the circuit using NGSPICE tools and obtained the results of Table \ref{tab:sim_op}.

%TABELA OP
\begin{table}[H]
	\centering
	\begin{tabular}{|l|r|}
		\hline    
		{\bf Name } & {\bf Value [V]} \\ \hline
		\input{../sim/oper_TAB.tex}
	\end{tabular}
	\caption{Simulation Operation Point Analysis.}
	\label{tab:sim_op}
\end{table}


\subsection{Gain, Cutoff Frequencies, Bandwidth and Merit Figure}

The Gain, lower and upper cutoff frequencies and bandwidth obtained are presented in Table \ref{tab:sim_results}, as well as the total cost of the components used in the circuit and the MERIT value. In Table \ref{tab:sim_results}, "up" states for Upper Cutoff Frequency and "low" for Lower Cutoff Frequency.

\begin{table}[H]
	\centering
	\begin{tabular}{|l|r|}
		\hline    
		{\bf Name} & {\bf Value} \\ \hline
		\input{../sim/op_TAB}
	\end{tabular}
	\caption{Main results from simulation analysis. The Upper and Lower Cutoff Frequencies and bandwidth are expressed in Hz. The cost is in MU (monetary units).}
	\label{tab:sim_results}
\end{table}

A human can ear frequencies between 20 Hz to 20 kHz. The Lower Cutoff Frequency obtained in simulation analysis is a little above 20 Hz, but very close to it. The Upper Cutoff Frequency meets the requirements.

\subsection{Input and Output Impedances}

The Table \ref{tab:sim_zin} shows the Input Impedance from simulation analysis.

\begin{table}[H]
	\centering
	\begin{tabular}{|l|r|}
		\hline    
		{\bf Name} & {\bf Value [kOhm]} \\ \hline
		\input{../sim/input_TAB}
	\end{tabular}
	\caption{Simulated Input Impedance.}
	\label{tab:sim_zin}
\end{table}

The Table \ref{tab:sim_zout} shows the Output Impedance from simulation analysis.

\begin{table}[H]
	\centering
	\begin{tabular}{|l|r|}
		\hline    
		{\bf Name} & {\bf Value [kOhm]} \\ \hline
		\input{../sim/output_TAB}
	\end{tabular}
	\caption{Simulated Output Impedance.}
	\label{tab:sim_zout}
\end{table}


\subsection{Output Voltage Gain in the Passband}
The result for the output voltage gain in the passband is shown in Figure \ref{fig:gain_sim}.

\begin{figure}[H] \centering
	\includegraphics[width=0.8\linewidth]{vo2f.eps}
	\caption{Output voltage gain in the passband.}
	\label{fig:gain_sim}
\end{figure}

\pagebreak

\subsection{Gain Frequency Response}

The gain in terms of frequency is shown in Figure \ref{fig:teo_gaindB}.


\begin{figure}[H] \centering
	\includegraphics[width=0.8\linewidth]{gain.eps}
	\caption {Frequency response of the gain in dB scale. The \textit{x axis} represents the frequency in Hertz (Hz) and the \textit{y axis} the gain in Volts.}
	\label{fig:teo_gaindB}
\end{figure}





