\newpage
\epstopdfsetup{outdir=./}
\section{Theoretical Analysis}
\label{sec:analysis}

In this section we will analyse the two stages of the circuit shown in Figure~\ref{fig:t3_circuit}, the gain stage and the output stage, theoretically using Octave tools. Both the OP analysis and the frequency analysis for the gain will be calculated.\par

\subsection{Gain Stage}

For this section we will analyse the part of circuit correspondent to the gain stage, in which was used the BC547A model which use the following values:


\begin{table}[H]
  \centering
  \begin{tabular}{|l|r|}
    \hline    
    {\bf Name [unit]} & {\bf Value} \\ \hline
    \input{../mat/ModelData_TAB}
  \end{tabular}
  \caption{BC547A model values.}
  \label{tab:teo}
\end{table}

And the used values of the circuit components used where the following:\par

\begin{table}[H]
	\centering
	\begin{tabular}{|l|r|}
		\hline    
		{\bf Name [unit]} & {\bf Value} \\ \hline
		\input{../mat/UsedValues_TAB}
	\end{tabular}
	\caption{Chosen values of the circuit components.}
	\label{tab:teo}
\end{table}


\subsubsection{Operating Point Analysis}

Doing an operating point analysis in the gain stage circuit, the following values are obtained, these values will be further compared to the results obtained using the NGspice in the conclusion.\par
 

\begin{table}[H]
	\centering
	\begin{tabular}{|l|r|}
		\hline    
		{\bf Name [unit]} & {\bf Value} \\ \hline
		\input{../mat/GainOP_TAB}
	\end{tabular}
	\caption{Gain Stage-Operating Point.}
	\label{tab:teo}
\end{table}

\subsubsection{Gain and Impedance results}

The input and output impedance of the gain stage are:\par

\begin{table}[H]
	\centering
	\begin{tabular}{|l|r|}
		\hline    
		{\bf Name [unit]} & {\bf Value} \\ \hline
		\input{../mat/GainImpedance_TAB}
	\end{tabular}
	\caption{Gain Stage Impedance values.}
	\label{tab:teo}
\end{table}

And the values for the AV in the gain stage where the following:

\begin{table}[H]
	\centering
	\begin{tabular}{|l|r|}
		\hline    
		{\bf Name [unit]} & {\bf Value} \\ \hline
		\input{../mat/GainAV_TAB}
	\end{tabular}
	\caption{Gain Stage AV values.}
	\label{tab:teo}
\end{table}

\subsection{Output Stage}

For this section we will analyse the part of circuit correspondent to the output stage, in which was used the following model, BC557A, which use these values:


\begin{table}[H]
	\centering
	\begin{tabular}{|l|r|}
		\hline    
		{\bf Name [unit]} & {\bf Value} \\ \hline
		\input{../mat/OutputModel_TAB}
	\end{tabular}
	\caption{Output model BC557A values.}
	\label{tab:teo}
\end{table}

\subsubsection{Operating Point Analysis}

Doing an operating point analysis in the output stage circuit, the following values are obtained, these values will be further compared to the results obtained using the NGspice in the conclusion.\par


\begin{table}[H]
	\centering
	\begin{tabular}{|l|r|}
		\hline    
		{\bf Name [unit]} & {\bf Value} \\ \hline
		\input{../mat/OutputOP_TAB}
	\end{tabular}
	\caption{Output Stage-Operating Point.}
	\label{tab:teo}
\end{table}

\subsubsection{Gain and Impedance results}

The input and output impedance of the output stage are:\par

\begin{table}[H]
	\centering
	\begin{tabular}{|l|r|}
		\hline    
		{\bf Name [unit]} & {\bf Value} \\ \hline
		\input{../mat/OutputImpedance_TAB}
	\end{tabular}
	\caption{Output Stage Impedance values.}
	\label{tab:teo}
\end{table}

And the values for the AV in the gain stage where the following:

\begin{table}[H]
	\centering
	\begin{tabular}{|l|r|}
		\hline    
		{\bf Name [unit]} & {\bf Value} \\ \hline
		\input{../mat/OutputAV_TAB}
	\end{tabular}
	\caption{Output Stage AV values.}
	\label{tab:teo}
\end{table}

\subsection{Final}

The total and final output values of the circuit are:\par

\begin{table}[H]
	\centering
	\begin{tabular}{|l|r|}
		\hline    
		{\bf Name [unit]} & {\bf Value} \\ \hline
		\input{../mat/Total_TAB}
	\end{tabular}
	\caption{Final and total values.}
	\label{tab:teo}
\end{table}

\subsection{Frequency response}

The frequency response of the Gain was obtained using the nodal method and the following matrix:

\begin{equation}
\begin{bmatrix}
RS+ZCI+RB & -RB& 0& 0& 0& 0& 0 \\ 
0 & -ZEB & -ro1 & ZEB+ro1+RC1 & -RC1&  0& 0\\ 
0 & rpi1*gm1 & 1 & 0 & 0 &  0& 0\\ 
0 & 0 & 0 & -RC1 & zpi2+RC1 & ZE2 & -ZE2 \\ 
0 & 0 &0  & 0 & -1-zpi2*gm2  & 1 & 0 \\
0 & 0 &0  & 0 & 0 & -ZE2 & ZE2+ZCO+RL\\ 
RB & RB+rpi1+ZEB & 0 & -ZEB & 0 & 0 & 0
\end{bmatrix}
\begin{bmatrix}
V_{1}\\ 
V_{2}\\ 
V_{3}\\ 
V_{5}\\ 
V_{6}\\ 
V_{7}\\ 
V_{8}
\end{bmatrix}
=
\begin{bmatrix}
V_{input}\\ 
0\\ 
0\\ 
0\\ 
0\\ 
0\\ 
0
\end{bmatrix}
\end{equation}


And the graphics obtained were the following, these will be compared with the graphics obtained using NGspice in the Conclusion Section of the report.

\begin{figure}[H] \centering
	\includegraphics[width=0.8\linewidth]{gain_octave.eps}
	\caption{Frequency response of the gain. The \textit{x axis} represents the frequency in Hertz (Hz) and the \textit{y axis} the gain in Volts.}
	\label{fig:teo_gain}
\end{figure}

\begin{figure}[H] \centering
	\includegraphics[width=0.8\linewidth]{gaindb_octave.eps}
	\caption {Frequency response of the gain in dB scale. The \textit{x axis} represents the frequency in Hertz (Hz) and the \textit{y axis} the gain in Volts.}
	\label{fig:teo_gaindB}
\end{figure}
