\newpage
\section{Comparison}
\label{sec:comparison}
\par
We will now compare the results obtained in the theoretical analysis and the simulation analysis.\par
\par
\begin{table}[h]
	\centering
	\begin{tabular}{|l|l|l|}
	\hline
	\textbf{Name} & \textbf{Theoretical} & \textbf{Simulated} \\ \hline
	Average  & 12.0000 V  & 12.0000 V  \\ \hline
	Ripple  & 0.000900 V & 0.008640 V  \\ \hline
	Cost  & 257.2 MU  & 257.2 MU  \\ \hline
	Merit  & 4.313424  & 0.4499508  \\ \hline
	\end{tabular}
	\caption{Results from simulation analysis.}
	\label{tab:valuessim}
\end{table}

The differences in the results are mainly seen in the value of the output voltage ripple, this difference is mainly due to the difference of the models of diodes and the complexity of the various components used both in Octave and Ngspice.\par

The not very great merit obtained in simulation analysis is due to a bigger than ideal output voltage ripple.

Despite the differences and low merit value on the Ngspice simulation the results achieved were the best found for the architecture chosen, after many iterations, these values for the various components were the best and the ones that gave us the best results, and most importantly the output voltage of 12 V.


\newpage
\section{Conclusion}
\label{sec:conclusion}

Given the comparison made previously and the results obtained along the work, we can say that the objectives for this laboratory assignment were achieved. In this lab assignment we were able to analyse, create and study a circuit that represents a AC/DC Converter.\par
Given all this the objectives for this laboratory assignment were all achieved.
