\section{Theoretical Analysis}
\label{sec:analysis}

In this section, the circuit shown in Figure~\ref{fig:t1} is analysed
theoretically using the mesh and nodal methods.

\subsection{Mesh Analysys}

The application of the mesh method in this circuit requires 4 currents to be determined. Each one of them will loop in one of the elementary meshes.
In order to facilitate the aplication of the method, each mesh was labeled with a letter, as shown in Figure~\ref{fig:t1}, and the designations $I_A$, $I_B$, $I_C$ and $I_D$
were assigned to the currents in meshes A, B, C and D, respectively. 

Applying the Kirchhoff's Voltage Law (KVL) to each of these meshes, the following equations are obtained:\vspace{4mm}

Mesh A:
\begin{equation}
  (R_1+R_3+R_4)I_A - R_3I_B - R_4I_C = -V_a
  \label{eq:A}
\end{equation}

Mesh B:
\begin{equation}
  -K_bR_3I_A + (K_bR_3-1)I_B = 0
  \label{eq:B}
\end{equation}

Mesh C:
\begin{equation}
  R_4I_A + (-R_6-R_7+K_c-R_4)I_C = 0
  \label{eq:C}
\end{equation}

Mesh D:
\begin{equation}
  I_D = I_d
  \label{eq:D}
\end{equation}

Solving this equations as a system of equations matrix in OCTAVE, the following values to the currents are obtained:

\begin{table}[h]
  \centering
  \begin{tabular}{|l|r|}
    \hline    
    {\bf Name} & {\bf Value [A]} \\ \hline
    \input{../mat/op_tab}
  \end{tabular}
  \caption{Mesh currents from mesh analysis.}
  \label{tab:meshI}
\end{table}

Now, applying the Ohm's Law, it's possible to obtain the node voltages. The following equations result of this application to each node: \vspace{4mm}

Node 1:
\begin{equation}
  V_1 = R_1I_A
  \label{eq:V1}
\end{equation}

Node 2:
\begin{equation}
  V_2 = V_1+R_2I_B
  \label{eq:V2}
\end{equation}

Node 3:
\begin{equation}
  V_3 = -V_a
  \label{eq:V3}
\end{equation}

Node 4:
\begin{equation}
  V_4 = V_1 - R_3(-I_A+I_B)
  \label{eq:V4}
\end{equation}

Node 5:
\begin{equation}
  V_5 = V_4 - R_5(I_B-I_D)
  \label{eq:V5}
\end{equation}

Node 6:
\begin{equation}
  V_6 = V_3 - R_6I_C
  \label{eq:V6}
\end{equation}

Node 8:
\begin{equation}
  V_8 = V_6 - R_7I_C
  \label{eq:V8}
\end{equation}

The Table~\ref{tab:meshV} shows the results from these equations:

\begin{table}[h]
  \centering
  \begin{tabular}{|l|r|}
    \hline    
    {\bf Name} & {\bf Value [V]} \\ \hline
    \input{../mat/op_tab1}
  \end{tabular}
  \caption{Node voltages from mesh analysis.}
  \label{tab:meshV}
\end{table}

The previous equations already included the relations necessary to determine the current of each component of the circuit using the currents of the meshes. As an exemple, the current of $R_3$ can be determined using the relation:

\begin{equation}
  I(R_3) = I_B - I_A
  \label{eq:IR3}
\end{equation}

\subsection{Node Analysys}

In order to theoretically analyse the circuit using the nodal method, 
8 nodes have been identified, as shown in Figure~\ref{fig:t1}. For 5 of these, the respective equations have been written. 
The node on the upper left corner was considered to have 0V. 
This way, its equation doesn't need to be written. 
The nodes 4 and 8, located on the terminals of the dependent voltage source, were grouped in a supernode in order to facilitate writing the equations. 
Also, the equation that describes the dependent voltage source was considered and an additional equation was written in node 3.

This way, a 7 equation and 7 unknowns system was determined and put in matrix form in order to solve it in Octave: \vspace{4mm}

$
\begin{bmatrix}
   -\frac{1}{R_2}-\frac{1}{R_3}-\frac{1}{R_1} & \frac{1}{R_2} & 0 & \frac{1}{R_3} & 0 & 0 & 0 \\
   K_b+\frac{1}{R_2} & -\frac{1}{R_2} & 0 & -K_b & 0 & 0 & 0 \\ 
   0 & 0 & 1 & 0 & 0 & 0 & 0 \\
   \frac{1}{R_3} & 0 & \frac{1}{R_4} & -\frac{1}{R_4}-\frac{1}{R_3}-\frac{1}{R_5} &  \frac{1}{R_5} & \frac{1}{R_7} & -\frac{1}{R_7} \\
   K_b & 0 & 0 & -\frac{1}{R_5}-K_b & \frac{1}{R_5} & 0 & 0 \\
   0 & 0 &\frac{1}{R_6} & 0 & 0 & -\frac{1}{R_6}-\frac{1}{R_7} & \frac{1}{R_7} \\
   0 & 0 & \frac{K_c}{R_6} & -1 & 0 & -\frac{K_c}{R_6} & 1 \\
\end{bmatrix}
\begin{bmatrix}
   V_1 \\
   V_2 \\
   V_3 \\
   V_4 \\
   V_5 \\
   V_6 \\
   V_8 \\
\end{bmatrix}
=
\begin{bmatrix}
   0 \\
   0 \\
   -V_a \\ 
   I_d \\
   I_d \\
   0 \\
   0 \\
\end{bmatrix}
$
\vspace{5mm}

Node equations determined using the Kirchhoff Current Law (KCL): \vspace{2mm}

Node 1:
\begin{equation}
  (-\frac{1}{R_2}-\frac{1}{R_3}-\frac{1}{R_1})V_1 + \frac{1}{R_2}V_2 + \frac{1}{R_3}V_4 = 0
  \label{eq:node1}
\end{equation}

Node 2:
\begin{equation}
  (K_b+\frac{1}{R_2})V_1 - \frac{1}{R_2}V_2 - K_bV_4 = 0
  \label{eq:node2}
\end{equation}

Supernode:
\begin{equation}
  \frac{1}{R_3}V_1 + \frac{1}{R_4}V_3 + (-\frac{1}{R_4}-\frac{1}{R_3}-\frac{1}{R_5})V_4 + \frac{1}{R_5}V_5 + \frac{1}{R_7}V_6 - \frac{1}{R_7}V_8 = I_d
  \label{eq:node4}
\end{equation}

Node 5:
\begin{equation}
  K_bV_1 + (-\frac{1}{R_5}-K_b)V_4 + \frac{1}{R_5}V_6 = I_d
  \label{eq:node5}
\end{equation}

Node 6:
\begin{equation}
  \frac{1}{R_6}V_3 + (-\frac{1}{R_6}-\frac{1}{R_7})V_6 + \frac{1}{R_7}V_8 = 0
  \label{eq:node6}
\end{equation}

\vspace{4mm}
Additional equations: \vspace{2mm}

\begin{equation}
  \frac{K_c}{R_6}V_3 - V_4 -\frac{K_c}{R_6}V_6 + V_8 = 0
  \label{eq:node8}
\end{equation}

\begin{equation}
  V_3 = -V_a
  \label{eq:node3}
\end{equation}

The results of the previous system of equations are shown in Table~\ref{tab:nodeV}:
 
\begin{table}[h]
  \centering
  \begin{tabular}{|l|r|}
    \hline    
    {\bf Name} & {\bf Value [V]} \\ \hline
    \input{../mat/op_tab2}
  \end{tabular}
  \caption{Node voltages from node analysis.}
  \label{tab:nodeV}
\end{table}

After finding the voltages of every node, Kirchoff's Voltage Law was applied in order to check the results. 
On the meshes A and D, the sum of all potential differences was, in fact, 0. 
On the meshes B and C, the result was, respectively, $-5.551115123125783e-17$ and $-8.881784197001252e-16$, which are both very close to 0.

The information of the node voltages is enough to determine the current of each component of the circuit, using the Ohm's Law. As as example, the following relations can be used to determine the current of $R_3$:

\begin{equation}
  V(R_3) = V_1 - V_4
  \label{eq:VR3n}
\end{equation}

\begin{equation}
  I(R_3) = \frac{V(R_3)}{R_3}
  \label{eq:IR3n}
\end{equation}


