\section{Conclusion}
\label{sec:conclusion}

In this laboratory assignment the objective of analysing a circuit using the Ngspice, and both the mesh and node theoretic methods has been achieved.

Static analysis has been performed both theoretically using the Octave maths tool and by circuit simulation using the Ngspice tool. The simulation results and the theoretical results were identical. The reason for this similarity is the fact that this is a straightforward and simple circuit containing only linear components, so the theoretical and simulation models don't differ. The static analysis done prevents also a lot of variables that can affect a circuit and it's components like time progression and other external and internal factors.

Also given the similarity between the two type of analysis is safe to assume that the Ngspice engine uses the same methods used theoretically, mainly the node analysis because it's easier the engine to identify the different nodes than to identify the meshes. So because of this, the values are, naturally, identical given that they are achieved using the same equations and methods.
All the values obtained for the various components and nodes follow the various laws that characterize a circuit, KVL and KCL with a error of 0\% or very close to it, and are what was expected given the values given to study the circuit in terms of grandness and direction of currents, which leads to believe that the values achieved are correct and the objective of this laboratory assignment was achieved.


